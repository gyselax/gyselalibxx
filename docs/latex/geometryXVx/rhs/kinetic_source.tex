
\documentclass[11pt]{article}

\usepackage[utf8]{inputenc}
\usepackage[T1]{fontenc}
\usepackage[english]{babel}
\usepackage{amsmath, amssymb}
\usepackage{biblatex}
\usepackage{csquotes}
\usepackage{graphics}
\usepackage{epsfig}
\usepackage{changepage}
\usepackage{amsmath}
\usepackage{amssymb}
\usepackage{amscd}
\usepackage{hyperref}
\usepackage{cleveref}
\addbibresource{library.bib}
\usepackage[margin=1.2in]{geometry}


\newcommand{\HRule}[1]{\rule{0.5\linewidth}{#1}} 

\begin{document}

\begin{center}
    \LARGE \textbf{Kinetic source} \\
    \vspace{0.8cm} 
\end{center}

The source term developed hereafter is inspired by the one currently implemented in GYSELA \cite[Appendix A]{Sarazin_2011}. It allows to inject independently density and energy. It has the following normalized expression
%
\begin{equation}\label{eq:general_form}
   S_\mathrm{sc}(x, v_a) = s_\mathrm{k} \dfrac{\mathcal{M}_\mathrm{sc}(x)}{\int_{0}^{L_x} \mathcal{M}_\mathrm{sc}(x) \, dx} S_\mathrm{v}(v_a) 
\end{equation}
%
The normalized velocity variable for species $a$ $v_a = v/v_{\mathrm{th}_a}$ is simply denoted by $v$ in the following. The mask function $\mathcal{M}_\mathrm{sc}$ defines the spatial extent of the source, most of the time it has a hyperbolic tangent expression, see the \texttt{mask\_tanh} documentation. The  $S_\mathrm{v}$ term is written as
%
\begin{equation}\label{eq:sv_expression}
  S_\mathrm{v}(v) = \left\{ s_0\left( \dfrac{3}{2} - \dfrac{v^2}{2 T_\mathrm{sc}} \right) + s_2 \left( -\dfrac{1}{2} + \dfrac{v^2}{2 T_\mathrm{sc}} \right)  \right\} \, \dfrac{1}{\sqrt{2 \pi  T_\mathrm{sc}} } \, e^{- \dfrac{v^2}{2 T_\mathrm{sc}}}
\end{equation}
%
In Eq.~\ref{eq:general_form} the $L_x$ term stands for the simulation box length. The $T_\mathrm{sc}$ parameter defines the source temperature, which is constant in space and time. The  $s_0$, $s_2$ and $s_\mathrm{k}$ are numerical inputs of the code that allows to define the properties of the source. In general we use $s_0 = 1$ so that the magnitude of the source is controlled by the parameter $s_\mathrm{k}$ alone. In this case we have indeed $\int_{}^{} d x \int_{}^{} dv \, S_\mathrm{sc} = s_\mathrm{k}$. Using $s_0 = s_2 = 1$ we recover a Maxwellian source, i.e.\ in this case 
%
\begin{equation}
  S_\mathrm{v}(v) = \dfrac{1}{\sqrt{2\pi T_\mathrm{sc}}} \, e^{- \dfrac{v^2}{2 T_\mathrm{sc}}}
\end{equation}
%
Conversely, by taking $s_2 = 0$ the expression of $S_\mathrm{v}$ has a vanishing first moment, that is to say in this particular case $\int_{}^{} dv\, S_\mathrm{v} = 0$, the source injects energy but no particles. Whatever the values given to $s_0$ and $s_2$ the source is symmetric with respect to $v=0$ therefore it does not inject any net momentum.

\section{Derivation of the source}
\label{sub:derivation_of_the_source_expression}

Let $S_\mathrm{v}(v)$ be a general source term. We decompose it on the Hermite polynomials basis. By doing this it becomes possible to adjust independently the amount of particles, momentum and energy that this source injects. For two functions $f$ and $g$ let us first introduce the scalar product
%
\begin{equation}
    \left< f,g\right> = \int_{-\infty}^{+\infty} f(y) g(y) e^{-y^2} dy
\end{equation}
%
The first three Hermite polynomials are written as
%
\begin{adjustwidth}{1.2cm}{1.2cm}
\begin{align*}
    &
    \begin{aligned}
        H_0 = 1
    \end{aligned} 
    &&
    \begin{aligned}
        \left| H_0 \right|^2 = \sqrt{\pi} 
    \end{aligned}\\
    &
    \begin{aligned}
        H_1 = 2X    
    \end{aligned} 
    &&
    \begin{aligned}
        \left| H_1 \right| ^2 = 2 \sqrt{\pi} 
    \end{aligned}\\
    &
    \begin{aligned}
        H_2 = - 2 + 4X^2
    \end{aligned}
    &&
    \begin{aligned}
        \left| H_2 \right| ^2 = 8 \sqrt{\pi} 
    \end{aligned}
\end{align*}
\end{adjustwidth}
%
The Hermite polynomials form an orthogonal basis for the scalar product defined above, i.e.\  $\left< H_h, H_h'\right> = \delta_{h,h'} |H_h|^2$. $\delta_{h,h'}$ is the Kronecker symbol that verifies $\delta_{h,h'} = 1$ if $h=h'$ and $\delta_{h,h'} = 0$ otherwise. Projecting the source term $S_\mathrm{v}$ on this basis gives
%
\begin{equation}\label{eq:source_expansion}
     S_{v}(v) = \sum\limits_{h = 0}^{+\infty} c_h \, H_h \left( \dfrac{v}{\sqrt{2 T_\mathrm{sc}} } \right) e^{- \dfrac{v^2}{2 T_\mathrm{sc}}}
\end{equation}
%
The $c_h$ terms are real valued coefficients. We also introduced here the source temperature $T_\mathrm{sc}$. We now take first three moments of this expansion to retrieve the particle, momentum and energy fluid sources associated with the $S_\mathrm{v}$ source term. Using the orthogonality of the Hermite basis we can show that these fluid sources are expressed as 
%
\begin{align*}
    &
    \begin{aligned}
        \int_{-\infty}^{+\infty} dv\, S_\mathrm{v}(v) = \sqrt{2 T_\mathrm{sc}} \, \sum\limits_{h}^{} \left<H_0, c_h H_h \right> = \sqrt{2 \pi  T_\mathrm{sc}} \, c_0
    \end{aligned} \\
    &
    \begin{aligned}
    \int_{-\infty}^{+\infty} dv\, v S_\mathrm{v}(v) = T_\mathrm{sc} \, \sum\limits_{h}^{} \left<H_1, c_h H_h \right> = 2\sqrt{\pi} T_\mathrm{sc} \, c_1
    \end{aligned} \\
    &
    \begin{aligned}
    \int_{-\infty}^{+\infty} dv\, \dfrac{1}{2} v^2 S_\mathrm{v}(v) =  \sqrt{2\pi } T_\mathrm{sc}^{3/2}  \left( 2c_2 + \dfrac{1}{2}c_0 \right)
    \end{aligned}
\end{align*}
%
Neglecting in the source expansion Eq.~\ref{eq:source_expansion} all the terms for $h>2$ we can obtain a source $\mathcal{S}_n$ that injects particles but no energy nor momentum by setting $c_1 = 0$ and $c_2 = -\tfrac{1}{4}c_0$. This source can be written as
%
\begin{equation}\label{eq:source_density_only}
\mathcal{S}_\mathrm{n}(v) = c_0 \left( \dfrac{3}{2} - \dfrac{v^2}{2 T_\mathrm{sc}} \right) e^{- \dfrac{v^2}{2 T_\mathrm{sc}}} 
\end{equation}
%
Conversely, a source that injects only momentum but no net particles nor energy is defined by $c_0 = 0$ and $c_2 = 0$. It is expressed as 
%
\begin{equation}\label{eq:source_momentum_only}
\mathcal{S}_\mathrm{u}(v) = c_1 \sqrt{ \dfrac{2}{T_\mathrm{sc}}}  e^{- \dfrac{v^2}{2 T_\mathrm{sc}}} 
\end{equation}
%
Lastly, a source that injects only energy can be obtained with $c_0 = 0$ and $c_1 = 0$. It is written as 
%
\begin{equation}\label{eq:source_energy_only}
\mathcal{S}_\mathrm{h}(v) = 2c_2 \left( -1 + \dfrac{v^2}{T_\mathrm{sc}} \right)   e^{- \dfrac{v^2}{2 T_\mathrm{sc}}} 
\end{equation}
%
We construct our source $S_\mathrm{v}$ using the three independent sources above as $S_\mathrm{v} = \mathcal{S}_\mathrm{n} + \mathcal{S}_\mathrm{u} + \mathcal{S}_\mathrm{h}$. By tuning the $c_0$, $c_1$ and $c_2$ parameter we can inject independently particles, momentum and energy in the plasma. In practice we always take $c_1=0$, i.e.\ the source does not inject any net momentum. Additionally by using $s_0 = \sqrt{2 \pi T_\mathrm{sc}}c_0 $ and $s_2 = \sqrt{2 \pi T_\mathrm{sc}}c_2 $ as input parameters in the code we are left with Eq.~\ref{eq:sv_expression} for the expression of $S_\mathrm{v}$.

\newpage
\printbibliography

\end{document}
