\documentclass[11pt]{article}

\usepackage[utf8]{inputenc}
\usepackage[T1]{fontenc}
\usepackage[english]{babel}
\usepackage{amsmath, amssymb}
\usepackage{biblatex}
\usepackage{csquotes}
\usepackage{graphics}
\usepackage{epsfig}
\usepackage{changepage}
\usepackage{amsmath}
\usepackage{amssymb}
\usepackage{amscd}
\usepackage{hyperref}
\usepackage{cleveref}
\addbibresource{library.bib}
\usepackage[margin=1.2in]{geometry}


\newcommand{\HRule}[1]{\rule{0.5\linewidth}{#1}} 

\begin{document}

\begin{center}
    \LARGE \textbf{Sheath equations} \\
    \vspace{0.8cm} 
\end{center}


\section{Equations describing plasma-wall interaction}
\paragraph{Boltzmann-Poisson system} The \texttt{sheath} executable solves the following Boltzmann-Poisson system
%
\begin{eqnarray}\label{eq:boltzmann_eq_ref}
  \partial_{t} f_a + v \partial_{x} f_a + \dfrac{q_a E}{m_a} \partial_{v} f_a = \mathcal{C}(f_a) + \mathcal{S}(f_a), \\[0.2cm]
  -\partial_{x}^2 \phi = \rho / \varepsilon_0 \quad \text{with} \quad \rho = \sum\limits_{\mathrm{species}}^{} q_a n_a.
\end{eqnarray}
%
Where we use the $x$ variable to specify the position. The $v$ variable expresses the velocity of the considered particle. $f_a(v,x,t)$ is the distribution function of species $a$, which expresses the density of particles at time $t$ and at point $(x,v)$ of phase space.  The subscript $a$ denotes the species the quantity refers to:---$\mathrm{e}$ for electrons, $\mathrm{i}$ for ions. The mass of any particle of species $a$ is written $m_a$, and its charge is $q_a$. $\varepsilon_0$ is the dielectric permittivity of vacuum. The electric field is written $E$, and we have $E = -\partial_{x} \phi$. We also introduced the local density $n_a$ of species $a$, which is a function of time and space. It is defined as the integral over the velocity space $n_a = \int_{}^{} dv \, f_a$. The $\mathcal{C}(f_a)$ term accounts for collisions, and the $\mathcal{S}(f_a)$ operator represents sources and sinks. Depending on the simulation parameters, these operators can be a combination of the terms described in the following. For more detail about each operator, see the corresponding documentation. \\

\noindent The Boltzmann-Poisson system is normalized using a reference density $n_0$ and temperature $T_0$. Time is normalized to the inverse of the electron plasma frequency $\omega_{\mathrm{pe_0}} = \sqrt{n_0 e^2 / m_\mathrm{e}\varepsilon_0} $. The space variable $x$ is normalized to the length scale relevant for plasma-wall interaction studies, i.e.\ the Debye length $\lambda_{\mathrm{D}_0} = \sqrt{\varepsilon_0 T_0 / n_0 e^2} $. The electrostatic potential is normalized to $T_0 / e$. The phase space velocity variable $v$ is normalized to the thermal velocity of each species $v_{T_0a}$ as $v_a = v/v_{T_0 a}$, with the thermal velocity of species $a$ written as $v_{T_0a} =  \sqrt{T_0/m_a}$. Particles distribution functions are normalized to the reference particle density in phase space $n_0 / v_{T_0a}$. We use the notation $A_a = m_\mathrm{e} / m_a$ to express the mass ratio between electrons and species $a$ (in particular $A_\mathrm{e} = 1$). The normalized charge of species $a$ is written as  $Z_a = q_a / e$. The normalized Boltzmann-Poisson system then reads as follows
%
\begin{eqnarray}
  && \partial_{t} f_a + \sqrt{A_a} \left( v_a \partial_{x} f_a \right. - \left. Z_a \partial_{x} \phi \, \partial_{v_a} f_a \right)   = \mathcal{C}(f_a) + \mathcal{S}(f_a), \label{eq:boltzmann_norm} \\[0.2cm]
  && -\partial_{x}^2 \phi = \rho \quad \text{with} \quad \rho = \sum\limits_{\mathrm{species}}^{} Z_a n_a. \label{eq:poisson_norm}
\end{eqnarray}
%
In all the following each quantity of interest is normalized. If a reference to some dimensional quantities is needed, it will be explicitly pointed out.
%
\paragraph{Sources and sinks} The source term $\mathcal{S}(f_a)$ can be composed of the following operators.

\begin{enumerate}
  \item A Bhatnagar-Gross-Krook operator \cite{Bhatnagar1954} of the form
  %
    \begin{equation}\label{eq:bgk}
    \mathcal{S}(f_a) = - \nu_{a}\mathcal{M}(x)  \left( f_a - g \right),
  \end{equation}
  %
that relaxes the distribution function $f_a$ towards a target function $g$. Depending on the characteristics of the target function this operator can inject or absorb particles. The $\mathcal{M}(x)$ term is a function of space that defines the simulation region where the operator becomes active. Typically this function has a hyperbolic tangent shape and transitions smoothly between a region where it is equal to zero---therefore where the BGK operator is not active---and another region where it is equal to one. $\nu_{a}$ is a coefficient that sets the operator magnitude. This coefficient can be either a constant, or can be adapted at each spatial position and each timestep to ensure that this BGK term conserves the charge locally.
%
\item A source term \cite[Appendix A]{Sarazin2011} defined by
%
\begin{equation}\label{eq:general_form}
   S(f_a) = s_\mathrm{k} \dfrac{\mathcal{M}(x)}{\int_{0}^{L_x} \mathcal{M}(x) \, dx} S_\mathrm{v}(v),
\end{equation}
%
where the mask function $\mathcal{M}$ defines the spatial extent of the source, similarly to its counterpart of Eq.~\ref{eq:bgk}. $L_x$ stands for the simulation box length. The  $S_\mathrm{v}$ term is written as
%
\begin{equation}\label{eq:sv_expression}
  S_\mathrm{v}(v) = \left\{ s_0\left( \dfrac{3}{2} - \dfrac{v^2}{2 T_\mathrm{s}} \right) + s_2 \left( -\dfrac{1}{2} + \dfrac{v^2}{2 T_\mathrm{s}} \right)  \right\} \, \dfrac{1}{\sqrt{2 \pi  T_\mathrm{s}} } \, e^{- \dfrac{v^2}{2 T_\mathrm{s}}}
\end{equation}
%
 The $T_\mathrm{s}$ is a constant parameter that defines the source temperature. The  $s_0$, $s_2$ and $s_\mathrm{k}$ parameters are numerical inputs of the code that define the properties of the source. In particular when using $s_0 = s_2 = 1$, the source adopts a Maxwellian shape whose magnitude is defined by the $s_\mathrm{k}$ parameter:
%
%
\begin{equation}
  S(f_a) = s_\mathrm{k} \dfrac{\mathcal{M}(x)}{\int_{0}^{L_x} \mathcal{M}(x) \, dx}\dfrac{1}{\sqrt{2\pi T_\mathrm{s}}} \, e^{- \dfrac{v^2}{2 T_\mathrm{s}}}.
\end{equation}
%
\end{enumerate}
%
\paragraph{Collisions} The collisions term $\mathcal{C}(f_a)$ accounts for both intra and inter species collisions. \\

\noindent An extensive description of the source, sink and collision operators described above can be found in the \texttt{src/geometryXVx/rhs/doc} folder. 
  \clearpage
  \printbibliography
\end{document}
