
\documentclass[11pt]{article}

\usepackage[utf8]{inputenc}
\usepackage[T1]{fontenc}
\usepackage[english]{babel}
\usepackage{amsmath, amssymb}
\usepackage{biblatex}
\usepackage{csquotes}
\usepackage{graphics}
\usepackage{epsfig}
\usepackage{changepage}
\usepackage{amsmath}
\usepackage{amssymb}
\usepackage{amscd}
\usepackage{hyperref}
\usepackage{cleveref}
\addbibresource{library.bib}
\usepackage[margin=1.2in]{geometry}


\newcommand{\HRule}[1]{\rule{0.5\linewidth}{#1}} 

\begin{document}

\begin{center}
    \LARGE \textbf{Hyperbolic mask function} \\
    \vspace{0.8cm} 
\end{center}

The hyperbolic mask denoted $\mathcal{M}(x)$ is a function that transitions smoothly between 0 and 1. It is mostly used to define some regions of the simulation box where specific operators become active. It is a hyperbolic tangent function that takes a value of one inside a specific domain, and vanishes outside. This mask can have two different forms. The first one is such that the mask equals one inside the domain  $\left[ x_\mathrm{\ell}, \, x_\mathrm{r} \right] $ and zero outside as
%
\begin{equation}\label{eq:mask_normal}
     \mathcal{M}(x) = \dfrac{1}{2}\left\{ \operatorname{tanh}{\left(\dfrac{x-x_\mathrm{\ell}}{d}\right)} - \operatorname{tanh}{\left(\dfrac{x-x_\mathrm{r}}{d}\right)}  \right\} .
\end{equation}
%
The $d$ parameter controls the steepness of the mask transition. The second possibility is to have a mask that is equal to zero inside the $\left[ x_\mathrm{\ell}, \, x_\mathrm{r} \right] $ domain, and zero outside. In this case it is expressed as
%
\begin{equation}\label{eq:mask_inverted}
     \mathcal{M}(x) = 1 - \dfrac{1}{2}\left\{ \operatorname{tanh}{\left(\dfrac{x-x_\mathrm{\ell}}{d}\right)} - \operatorname{tanh}{\left(\dfrac{x-x_\mathrm{r}}{d}\right)}  \right\} .
\end{equation}
%
Lastly, it is possible to use a normalized version of these masks function so that their integral is equal to one. For instance Eq.~(\ref{eq:mask_normal}) becomes  
%
\begin{equation*}\label{eq:mask_nmz}
     \mathcal{M}(x) = \dfrac{1}{2V_\mathcal{M}}\left\{ \operatorname{tanh}{\left(\dfrac{x-x_\mathrm{\ell}}{d}\right)} - \operatorname{tanh}{\left(\dfrac{x-x_\mathrm{r}}{d}\right)}  \right\},
\end{equation*}
%
with $V_\mathcal{M} = \int_{0}^{L_x} d x \, \mathcal{M}$, and $L_x$ is the total length of the simulation box.

\end{document}
